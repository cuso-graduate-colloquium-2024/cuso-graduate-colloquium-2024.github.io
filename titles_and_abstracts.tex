Speaker: Elia Bubani

Title: The sub-Riemannian affine-additive group.

Abstract:
The affine-additive group $\mathcal{AA}$ consists of $\mathbb{R}\times\mathcal{R}$, where $\mathbb{R}$ is the real line and $\mathcal{R}$ is the hyperbolic right half-plane, endowed with a natural group law. We consider $\mathcal{AA}$ as a metric measure space with a canonical left-invariant measure and a left-invariant sub-Riemannian metric. Throughout the talk we shall cover notions of parabolicity and hyperbolicity for metric measure spaces and then point out implications in the theory of quasiconformal mappings. As a consequence we obtain that $\mathcal{AA}$ is not quasiconformally equivalent to the sub-Riemannian Heisenberg group $\mathbb{H}$. If time allows we will discuss further relations between $\mathcal{AA}$ and $\mathbb{H}$ under the more general theory of quasiregular mappings.\\
Joint work with Z. Balogh and G. Platis.


Speaker: Steffan Polzer

Title: Phase Transition in the Infrared-Divergent Spin-Boson Model

Abstract: The spin-boson model describes the interaction between a quantum mechanical two level system and a bosonic field. Its Hamiltonian, a self-adjoint and lower-bounded operator, is said to have a ground state if the infimum of its spectrum is an eigenvalue. I present recent work in which we show that, in the infrared-divergent model, a phase transition occurs: as the coupling strength increases, the system transitions from having a ground state to having none. Along the way we will get to know both the Feynman-Kac formula and how it allows us to translate the functional analytic problem into the language of probability theory, as well as some important models and tools of mathematical statistical mechanics. Based on joint work with Volker Betz, Benjamin Hinrichs and Mino Nicola Kraft. 


Speaker: Marco Inversi

Title: Euler, hypersurfaces and other amenities

Abstract: The incompressible Euler system is a universal model describing conservation of momentum in an incompressible ideal fluid, i.e. a system where particles cannot compress and do not experience any internal friction. Despite being a closed system, it is experimentally and numerically checked that the kinetic energy gets dissipated along the time evolution. We investigate qualitative properties of the geometric structure of the set where the anomalous inviscid dissipation accumulates.


Speaker: Nicola Paddeu

Title: Regularity of sub-Riemannian geodesics

Abstract: We begin by introducing sub-Riemannian manifolds, presenting definitions and some basic properties. Next, we provide several motivating examples that illustrate the importance of studying these manifolds. Finally, we state and prove a regularity result for length-minimizing curves.


Speaker: Tobias Bisang

Title: Zeroes of Polynomials in $\mathbb{Q}^n$

Abstract: Given some $\mathbb{Q}$-polynomials, do they have a common zero in $\mathbb{Q}^n$? This talks shows how to check this for simple polynomials, using the $p$-adic norms and the Hasse-Minkowski theorem. The question is harder to answer for elliptic curves which provides motivation to study them.


Speaker: Denis Marti

Title: Lipschitz-volume rigidiy

Abstract: In this talk we consider the following question. Given a surjective $1$-Lipschitz map between two metric spaces, under what additional assumptions can we conclude that $f$ is an isometry? A classical result from Riemannian geometry states that this is the case if the target and domain are closed, oriented Riemannian manifolds of the same dimension and volume. However, simple examples show that this is false without any topological assumptions. For this reason, there has been recent interest in considering manifolds with less differentiable structure.
We first introduce the question and some techniques in the smooth setting. We then give a brief overview of further developments and recent work.


Speaker: Matteo Nesi
Title: Uniqueness for the Continuity Equation in Bounded Domains
Abstract: After a brief introduction to the continuity equation, we discuss the uniqueness of solutions to the corresponding Cauchy problem. We will see that the presence of a boundary will introduce some difficulties in the non-smooth setting, already at the level of defining weak solutions. 
Finally we present the two notions of normal trace that are needed and, time permitting, talk about their link to other areas of analysis.


Speaker: Jasmin Jörg
Title: Perturbed bouquets and their realisability by systoles

Abstract: Given a closed hyperbolic surface, one may consider the collection of all systoles, that is, all geodesics of minimal length. Systoles necessarily form a 1-system, a system of simple closed curves intersecting pairwise at most once. On the other hand, it is not clear what 1-systems can be realised by systoles. We consider the subclass of 1-systems obtained by perturbing a bouquet, present some obstructions to them being systolic and indicate why most of them fail to be systoles for any Riemannian metric.

Speaker: Dylan Müller
Title : "Spectral zeta functions on cyclic groups."
Abstract : "Finite cyclic groups, when viewed as compact subgroups embedded in the circle, exibit a form of self-duality with respect to the Fourier transform, though becoming discrete in the spectrum. Heuristically, these compact cyclic groups asymptotically approximate the circle, while their duals discrete copies correspond to approximation of the integers. Using heat kernels and spectral zeta functions, this relationship can be made rigorous, revealling interesting onnections to number theory."


